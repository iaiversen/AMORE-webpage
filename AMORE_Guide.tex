% Options for packages loaded elsewhere
\PassOptionsToPackage{unicode}{hyperref}
\PassOptionsToPackage{hyphens}{url}
\PassOptionsToPackage{dvipsnames,svgnames,x11names}{xcolor}
%
\documentclass[
  letterpaper,
  DIV=11,
  numbers=noendperiod]{scrartcl}

\usepackage{amsmath,amssymb}
\usepackage{iftex}
\ifPDFTeX
  \usepackage[T1]{fontenc}
  \usepackage[utf8]{inputenc}
  \usepackage{textcomp} % provide euro and other symbols
\else % if luatex or xetex
  \usepackage{unicode-math}
  \defaultfontfeatures{Scale=MatchLowercase}
  \defaultfontfeatures[\rmfamily]{Ligatures=TeX,Scale=1}
\fi
\usepackage{lmodern}
\ifPDFTeX\else  
    % xetex/luatex font selection
\fi
% Use upquote if available, for straight quotes in verbatim environments
\IfFileExists{upquote.sty}{\usepackage{upquote}}{}
\IfFileExists{microtype.sty}{% use microtype if available
  \usepackage[]{microtype}
  \UseMicrotypeSet[protrusion]{basicmath} % disable protrusion for tt fonts
}{}
\makeatletter
\@ifundefined{KOMAClassName}{% if non-KOMA class
  \IfFileExists{parskip.sty}{%
    \usepackage{parskip}
  }{% else
    \setlength{\parindent}{0pt}
    \setlength{\parskip}{6pt plus 2pt minus 1pt}}
}{% if KOMA class
  \KOMAoptions{parskip=half}}
\makeatother
\usepackage{xcolor}
\setlength{\emergencystretch}{3em} % prevent overfull lines
\setcounter{secnumdepth}{-\maxdimen} % remove section numbering
% Make \paragraph and \subparagraph free-standing
\makeatletter
\ifx\paragraph\undefined\else
  \let\oldparagraph\paragraph
  \renewcommand{\paragraph}{
    \@ifstar
      \xxxParagraphStar
      \xxxParagraphNoStar
  }
  \newcommand{\xxxParagraphStar}[1]{\oldparagraph*{#1}\mbox{}}
  \newcommand{\xxxParagraphNoStar}[1]{\oldparagraph{#1}\mbox{}}
\fi
\ifx\subparagraph\undefined\else
  \let\oldsubparagraph\subparagraph
  \renewcommand{\subparagraph}{
    \@ifstar
      \xxxSubParagraphStar
      \xxxSubParagraphNoStar
  }
  \newcommand{\xxxSubParagraphStar}[1]{\oldsubparagraph*{#1}\mbox{}}
  \newcommand{\xxxSubParagraphNoStar}[1]{\oldsubparagraph{#1}\mbox{}}
\fi
\makeatother

\usepackage{color}
\usepackage{fancyvrb}
\newcommand{\VerbBar}{|}
\newcommand{\VERB}{\Verb[commandchars=\\\{\}]}
\DefineVerbatimEnvironment{Highlighting}{Verbatim}{commandchars=\\\{\}}
% Add ',fontsize=\small' for more characters per line
\usepackage{framed}
\definecolor{shadecolor}{RGB}{241,243,245}
\newenvironment{Shaded}{\begin{snugshade}}{\end{snugshade}}
\newcommand{\AlertTok}[1]{\textcolor[rgb]{0.68,0.00,0.00}{#1}}
\newcommand{\AnnotationTok}[1]{\textcolor[rgb]{0.37,0.37,0.37}{#1}}
\newcommand{\AttributeTok}[1]{\textcolor[rgb]{0.40,0.45,0.13}{#1}}
\newcommand{\BaseNTok}[1]{\textcolor[rgb]{0.68,0.00,0.00}{#1}}
\newcommand{\BuiltInTok}[1]{\textcolor[rgb]{0.00,0.23,0.31}{#1}}
\newcommand{\CharTok}[1]{\textcolor[rgb]{0.13,0.47,0.30}{#1}}
\newcommand{\CommentTok}[1]{\textcolor[rgb]{0.37,0.37,0.37}{#1}}
\newcommand{\CommentVarTok}[1]{\textcolor[rgb]{0.37,0.37,0.37}{\textit{#1}}}
\newcommand{\ConstantTok}[1]{\textcolor[rgb]{0.56,0.35,0.01}{#1}}
\newcommand{\ControlFlowTok}[1]{\textcolor[rgb]{0.00,0.23,0.31}{\textbf{#1}}}
\newcommand{\DataTypeTok}[1]{\textcolor[rgb]{0.68,0.00,0.00}{#1}}
\newcommand{\DecValTok}[1]{\textcolor[rgb]{0.68,0.00,0.00}{#1}}
\newcommand{\DocumentationTok}[1]{\textcolor[rgb]{0.37,0.37,0.37}{\textit{#1}}}
\newcommand{\ErrorTok}[1]{\textcolor[rgb]{0.68,0.00,0.00}{#1}}
\newcommand{\ExtensionTok}[1]{\textcolor[rgb]{0.00,0.23,0.31}{#1}}
\newcommand{\FloatTok}[1]{\textcolor[rgb]{0.68,0.00,0.00}{#1}}
\newcommand{\FunctionTok}[1]{\textcolor[rgb]{0.28,0.35,0.67}{#1}}
\newcommand{\ImportTok}[1]{\textcolor[rgb]{0.00,0.46,0.62}{#1}}
\newcommand{\InformationTok}[1]{\textcolor[rgb]{0.37,0.37,0.37}{#1}}
\newcommand{\KeywordTok}[1]{\textcolor[rgb]{0.00,0.23,0.31}{\textbf{#1}}}
\newcommand{\NormalTok}[1]{\textcolor[rgb]{0.00,0.23,0.31}{#1}}
\newcommand{\OperatorTok}[1]{\textcolor[rgb]{0.37,0.37,0.37}{#1}}
\newcommand{\OtherTok}[1]{\textcolor[rgb]{0.00,0.23,0.31}{#1}}
\newcommand{\PreprocessorTok}[1]{\textcolor[rgb]{0.68,0.00,0.00}{#1}}
\newcommand{\RegionMarkerTok}[1]{\textcolor[rgb]{0.00,0.23,0.31}{#1}}
\newcommand{\SpecialCharTok}[1]{\textcolor[rgb]{0.37,0.37,0.37}{#1}}
\newcommand{\SpecialStringTok}[1]{\textcolor[rgb]{0.13,0.47,0.30}{#1}}
\newcommand{\StringTok}[1]{\textcolor[rgb]{0.13,0.47,0.30}{#1}}
\newcommand{\VariableTok}[1]{\textcolor[rgb]{0.07,0.07,0.07}{#1}}
\newcommand{\VerbatimStringTok}[1]{\textcolor[rgb]{0.13,0.47,0.30}{#1}}
\newcommand{\WarningTok}[1]{\textcolor[rgb]{0.37,0.37,0.37}{\textit{#1}}}

\providecommand{\tightlist}{%
  \setlength{\itemsep}{0pt}\setlength{\parskip}{0pt}}\usepackage{longtable,booktabs,array}
\usepackage{calc} % for calculating minipage widths
% Correct order of tables after \paragraph or \subparagraph
\usepackage{etoolbox}
\makeatletter
\patchcmd\longtable{\par}{\if@noskipsec\mbox{}\fi\par}{}{}
\makeatother
% Allow footnotes in longtable head/foot
\IfFileExists{footnotehyper.sty}{\usepackage{footnotehyper}}{\usepackage{footnote}}
\makesavenoteenv{longtable}
\usepackage{graphicx}
\makeatletter
\newsavebox\pandoc@box
\newcommand*\pandocbounded[1]{% scales image to fit in text height/width
  \sbox\pandoc@box{#1}%
  \Gscale@div\@tempa{\textheight}{\dimexpr\ht\pandoc@box+\dp\pandoc@box\relax}%
  \Gscale@div\@tempb{\linewidth}{\wd\pandoc@box}%
  \ifdim\@tempb\p@<\@tempa\p@\let\@tempa\@tempb\fi% select the smaller of both
  \ifdim\@tempa\p@<\p@\scalebox{\@tempa}{\usebox\pandoc@box}%
  \else\usebox{\pandoc@box}%
  \fi%
}
% Set default figure placement to htbp
\def\fps@figure{htbp}
\makeatother

\KOMAoption{captions}{tableheading}
\makeatletter
\@ifpackageloaded{caption}{}{\usepackage{caption}}
\AtBeginDocument{%
\ifdefined\contentsname
  \renewcommand*\contentsname{Table of contents}
\else
  \newcommand\contentsname{Table of contents}
\fi
\ifdefined\listfigurename
  \renewcommand*\listfigurename{List of Figures}
\else
  \newcommand\listfigurename{List of Figures}
\fi
\ifdefined\listtablename
  \renewcommand*\listtablename{List of Tables}
\else
  \newcommand\listtablename{List of Tables}
\fi
\ifdefined\figurename
  \renewcommand*\figurename{Figure}
\else
  \newcommand\figurename{Figure}
\fi
\ifdefined\tablename
  \renewcommand*\tablename{Table}
\else
  \newcommand\tablename{Table}
\fi
}
\@ifpackageloaded{float}{}{\usepackage{float}}
\floatstyle{ruled}
\@ifundefined{c@chapter}{\newfloat{codelisting}{h}{lop}}{\newfloat{codelisting}{h}{lop}[chapter]}
\floatname{codelisting}{Listing}
\newcommand*\listoflistings{\listof{codelisting}{List of Listings}}
\makeatother
\makeatletter
\makeatother
\makeatletter
\@ifpackageloaded{caption}{}{\usepackage{caption}}
\@ifpackageloaded{subcaption}{}{\usepackage{subcaption}}
\makeatother

\usepackage{bookmark}

\IfFileExists{xurl.sty}{\usepackage{xurl}}{} % add URL line breaks if available
\urlstyle{same} % disable monospaced font for URLs
\hypersetup{
  pdftitle={Guide for managing AMORE website},
  pdfauthor={Ingebjørg A. Iversen},
  colorlinks=true,
  linkcolor={blue},
  filecolor={Maroon},
  citecolor={Blue},
  urlcolor={Blue},
  pdfcreator={LaTeX via pandoc}}


\title{Guide for managing AMORE website}
\author{Ingebjørg A. Iversen}
\date{}

\begin{document}
\maketitle


\subsubsection{Table of content}\label{table-of-content}

\begin{enumerate}
\def\labelenumi{\arabic{enumi}.}
\tightlist
\item
  Important information
\item
  Getting Started (Git clone, dependencies)
\item
  Project Structure Overview

  \begin{enumerate}
  \def\labelenumii{\arabic{enumii}.}
  \item
    (Folder structure,
  \item
    what needs to be where, where do you need to save things to add it)
  \item
    Explaining the SCSS system, configurations files,
  \end{enumerate}
\item
  File-by-File Breakdown

  \begin{enumerate}
  \def\labelenumii{\arabic{enumii}.}
  \tightlist
  \item
    explain .gitignore
  \item
    Styles.scss

    \begin{enumerate}
    \def\labelenumiii{\arabic{enumiii}.}
    \tightlist
    \item
      The styles.scss is devided into different sections
    \item
      The standard screen size codes, and the @media sections with
      adjusted screen layout for smaller tablets and mobile screens
    \item
    \end{enumerate}
  \end{enumerate}
\item
  Troubleshooting (inspect in browser, rendering before pushing to git,
  sometimes invisible spaces ruing everything!!!!!)
\item
  Contributions Guidelines

  \begin{enumerate}
  \def\labelenumii{\arabic{enumii}.}
  \tightlist
  \item
    troubleshooting
  \item
    How to add project files
  \item
    how to add new website pages
  \item
    How to make design changes

    \begin{itemize}
    \item
      sync\_styles function (the function was created in the Setup.R
      script and syncs all the scss files across folders. These folders
      are there because quarto look for scss in different folders for
      different qmd files, and while one scss in the root folder
      technically should be enough, I experienced that quarto had
      problems with collecting or findings the scss rules unless I had
      the scss file in the same folder as the .qmd. However, some times
      even though there is a scss in the same folder as a .qmd the .qmd
      still gets its scss rules from another scss file, and it is no
      information about what file get rendering rules from what scss,
      therefore the scss file is distributed in all folders that need
      scss rendering rules, but are identical and synchronized from the
      root styles.scss. After you have done changes to the styles.scss
      in the root folder (AMORE-webpage'') write sync\_styles in the
      console (make sure the function has been activated from the
      Setup.R script).

      \begin{itemize}
      \tightlist
      \item
        if you make alterations, make sure those alterations fit with
        @media adjustable screen sizes
      \end{itemize}
    \item
      How to add new website pages (add to yaml navigation) save in
      correct folder
    \end{itemize}
  \end{enumerate}
\end{enumerate}

\subsection{1. Important information}\label{important-information}

\subsubsection{1.1 The Different
Languages}\label{the-different-languages}

The AMORE website is built using multiple programming languages and
frameworks that work together:

Quarto Markdown (.qmd): Primary content files for pages SCSS/CSS:
Styling and responsive design JavaScript: Interactive functionality and
client-side behavior R: Shiny app backend, data processing, and setup
scripts HTML: Embedded within Quarto files for custom components Shell
scripts: Deployment and automation tasks

\paragraph{Why multiple languages? Each serves a specific
purpose:}\label{why-multiple-languages-each-serves-a-specific-purpose}

Quarto allows academic content with citations and technical writing SCSS
provides maintainable, organized styling with variables JavaScript
enables interactivity (filter tabs, pagination) R powers the dynamic
Living Meta-Analysis directory HTML gives precise control over structure
when needed

\subsubsection{1.2 RStudio and Quarto}\label{rstudio-and-quarto}

RStudio is your primary development environment. Install:

R (version 4.0.0+) from CRAN RStudio Desktop from Posit Quarto CLI -
Usually bundled with RStudio, or install from quarto.org

Key RStudio features for this project:

Visual editor for .qmd files Terminal for git commands Console for
running R functions (like sync\_styles()) Files pane for navigation
Viewer for previewing rendered pages

\subsubsection{1.3 GitHub, ShinyApps.io, and
Netlify}\label{github-shinyapps.io-and-netlify}

Three platforms work together: GitHub (Version Control \& Collaboration)

Repository: iaiversen/AMORE-webpage Stores all source code Tracks
changes with commit history Enables collaboration through pull requests
Important: Never commit sensitive data (API keys, passwords)

ShinyApps.io (Shiny App Hosting)

Hosts the Living Meta-Analysis directory (app.R) Free tier limitations:
limited active hours, connection timeouts URL:
https://meta-oxytocin.shinyapps.io/shiny-meta/ Embedded in
Living\_meta-analysis\_Directory.qmd via iframe

Netlify (Website Hosting)

Automatically deploys from GitHub Builds site from \_site folder after
Quarto render Domain: amore-project.org Configuration in netlify.toml

Deployment workflow:

Make changes locally in RStudio Test by rendering (quarto render or
Render button) Commit and push to GitHub Netlify automatically rebuilds
and deploys For Shiny app: Deploy separately using
rsconnect::deployApp()

\subsubsection{1.4 Two-Factor
Authentication}\label{two-factor-authentication}

Why it matters: Protects the project from unauthorized access. Where you
need 2FA:

Netlify account: For deployment settings

\subsection{2. Getting started}\label{getting-started}

\subsubsection{2.1 Git Clone}\label{git-clone}

\textbf{First time setup:}

bash or Windows powershell

\begin{Shaded}
\begin{Highlighting}[]
\CommentTok{\# In terminal/command line:}
\BuiltInTok{cd}\NormalTok{ \textasciitilde{}/Documents  }\CommentTok{\# or your preferred location}
\FunctionTok{git}\NormalTok{ clone https://github.com/iaiversen/AMORE{-}webpage.git}
\BuiltInTok{cd}\NormalTok{ AMORE{-}webpage}
\end{Highlighting}
\end{Shaded}

\textbf{In RStudio:}

\begin{enumerate}
\def\labelenumi{\arabic{enumi}.}
\item
  File → New Project → Version Control → Git
\item
  Repository URL:
  \texttt{https://github.com/iaiversen/AMORE-webpage.git}
\item
  Choose directory location
\item
  Create Project
\end{enumerate}

\subsubsection{2.2 Dependencies}\label{dependencies}

\textbf{Install all required packages:}

Open \texttt{Setup.R} and run the entire script. This installs:

\textbf{Core packages:}

\begin{itemize}
\item
  \texttt{rmarkdown}, \texttt{knitr}, \texttt{quarto} - Document
  rendering
\item
  \texttt{shiny}, \texttt{rsconnect} - Shiny app deployment
\item
  \texttt{DT}, \texttt{yaml}, \texttt{fs}, \texttt{httr},
  \texttt{jsonlite} - Data handling
\item
  \texttt{bslib}, \texttt{sass} - Styling
\item
  \texttt{tinytex} - LaTeX/PDF support
\end{itemize}

\textbf{The script handles:}

\begin{itemize}
\item
  Checking if packages are already installed
\item
  Installing missing packages
\item
  Loading libraries
\item
  Setting up TinyTeX distribution
\end{itemize}

\textbf{Run once after cloning:}

\begin{Shaded}
\begin{Highlighting}[]
\FunctionTok{source}\NormalTok{(}\StringTok{"Setup.R"}\NormalTok{)}
\end{Highlighting}
\end{Shaded}

\subsubsection{2.3 Verify Installation}\label{verify-installation}

\textbf{Test that everything works:}

r console

\begin{Shaded}
\begin{Highlighting}[]
\NormalTok{(}\StringTok{"quarto {-}{-}version"}\NormalTok{) }\CommentTok{\# Check Quarto system}
\NormalTok{quarto}\SpecialCharTok{::}\FunctionTok{quarto\_render}\NormalTok{(}\StringTok{"index.qmd"}\NormalTok{) }\CommentTok{\# Test rendering a single page }
\FunctionTok{source}\NormalTok{(}\StringTok{"Setup.R"}\NormalTok{) }\FunctionTok{sync\_styles}\NormalTok{() }\CommentTok{\# Test sync\_styles function}
\end{Highlighting}
\end{Shaded}

\subsection{3. Project structure
overview}\label{project-structure-overview}

\begin{Shaded}
\begin{Highlighting}[]


\ExtensionTok{AMORE{-}webpage/}
\ExtensionTok{├──}\NormalTok{ .git/                      }\CommentTok{\# Git version control (don\textquotesingle{}t edit)}
\ExtensionTok{├──}\NormalTok{ .quarto/                   }\CommentTok{\# Quarto cache (don\textquotesingle{}t edit)}
\ExtensionTok{├──}\NormalTok{ \_site/                     }\CommentTok{\# Generated website (don\textquotesingle{}t commit)}
\ExtensionTok{├──}\NormalTok{ LMAs/                      }\CommentTok{\# Living Meta{-}Analysis project pages}
\ExtensionTok{│}\NormalTok{   ├── executive{-}function.qmd}
\ExtensionTok{│}\NormalTok{   ├── styles\_for\_lma.scss}
\ExtensionTok{│}\NormalTok{   └── lma{-}template.qmd}
\ExtensionTok{├──}\NormalTok{ shiny{-}meta/               }\CommentTok{\# Shiny app folder}
\ExtensionTok{│}\NormalTok{   └── app.R}
\ExtensionTok{├──}\NormalTok{ images and visualizations/ }\CommentTok{\# Assets}
\ExtensionTok{├──}\NormalTok{ \_quarto.yml               }\CommentTok{\# Site configuration}
\ExtensionTok{├──}\NormalTok{ styles.scss               }\CommentTok{\# Main stylesheet}
\ExtensionTok{├──}\NormalTok{ index.qmd                 }\CommentTok{\# Homepage}
\ExtensionTok{├──}\NormalTok{ about.qmd                 }\CommentTok{\# About page}
\ExtensionTok{├──}\NormalTok{ contact.qmd               }\CommentTok{\# Contact page}
\ExtensionTok{├──}\NormalTok{ Resources.qmd             }\CommentTok{\# Resources page}
\ExtensionTok{├──}\NormalTok{ guidelines.qmd            }\CommentTok{\# Guidelines page}
\ExtensionTok{├──}\NormalTok{ Standardization.qmd       }\CommentTok{\# Standards page}
\ExtensionTok{├──}\NormalTok{ Living\_meta{-}analysis\_Directory.qmd  }\CommentTok{\# Shiny embed}
\ExtensionTok{├──}\NormalTok{ Protocol\_checklist.qmd    }\CommentTok{\# Checklist tool}
\ExtensionTok{├──}\NormalTok{ Setup.R                   }\CommentTok{\# Dependency installer}
\ExtensionTok{├──}\NormalTok{ LICENSE                   }\CommentTok{\# MIT License}
\ExtensionTok{├──}\NormalTok{ .gitignore               }\CommentTok{\# Git exclusions}
\ExtensionTok{├──}\NormalTok{ netlify.toml             }\CommentTok{\# Netlify config}
\ExtensionTok{└──}\NormalTok{ README.md                }\CommentTok{\# GitHub readme}
\end{Highlighting}
\end{Shaded}

\subsection{4. File-by-file breakdown}\label{file-by-file-breakdown}

\subsubsection{4.1 .gitignore Explained}\label{gitignore-explained}

\textbf{Purpose:} Tells Git which files to ignore (not track/commit).

\textbf{Current exclusions:}

\begin{Shaded}
\begin{Highlighting}[]
\ExtensionTok{.Rproj.user/}      \CommentTok{\# RStudio project files}
\ExtensionTok{.Rhistory}         \CommentTok{\# R command history}
\ExtensionTok{.RData}            \CommentTok{\# R workspace data}
\ExtensionTok{.Ruserdata}        \CommentTok{\# User{-}specific R settings}
\ExtensionTok{/\_quarto/}         \CommentTok{\# Quarto cache}
\ExtensionTok{**/\_quarto\_internal\_scss\_error.scss}  \CommentTok{\# Temporary error files}
\end{Highlighting}
\end{Shaded}

\textbf{Why exclude these:}

\begin{itemize}
\item
  RStudio files are user-specific
\item
  \_site/ is generated, not source code
\item
  Cache files speed up local rendering but aren't needed in repo
\item
  Reduces repository size
\end{itemize}

\textbf{Never add to .gitignore:}

\begin{itemize}
\item
  Source files (.qmd, .R, .scss)
\item
  Configuration files (\_quarto.yml, netlify.toml)
\item
  Assets (images, logos)
\end{itemize}

\subsubsection{4.2 styles.scss Deep Dive}\label{styles.scss-deep-dive}




\end{document}
